Automated argumentation mining requires an adequate type system or annotation scheme for classifying the patterns of argument that succeed or fail in a corpus of legal documents. Moreover, there must be a reliable and accurate method for classifying the arguments found in natural language legal documents. Without an adequate and operational type system, we are unlikely to reach consensus on argument corpora that can function as a gold standard. This paper reports the preliminary results of research to annotate a sample of representative judicial decisions for the reasoning of the factfinder. The decisions report whether the evidence adduced by the petitioner adequately supports the claim that a medical theory causally links some type of vaccine with various types of injuries or adverse medical conditions. This paper summarizes and discusses some patterns of reasoning that we are finding, using examples from the corpus. The pattern types and examples presented here demonstrate the difficulty of developing a type or annotation system for characterizing the logically important patterns of reasoning.
