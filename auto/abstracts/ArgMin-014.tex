The ability to analyze the adequacy of supporting information is necessary for determining the strength of an argument. This is especially the case for online user comments, which often consist of arguments lacking proper substantiation and reasoning. Thus, we develop a framework for automatically classifying each proposition as UNVERIFIABLE, VERIFIABLE NONEXPERIENTIAL, or VERIFIABLE EXPERIENTIAL, where the appropriate type of support is reason, evidence, and optional evidence, respectively. Once the existing support for propositions are identified, this classification can provide an estimate of how adequately the arguments have been supported. We build a gold standard dataset of 9,476 sentences and clauses from 1,047 comments submitted to an eRulemaking platform and find that Support Vector Machine (SVM) classifiers trained with n-grams and additional features capturing the verifiability and experientiality exhibit statistically significant improvement over the unigram baseline, achieving a macro-averaged F1 of 68.99\%.
