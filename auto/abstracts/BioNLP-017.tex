Zoonotic viruses, viruses that are trans-mittable between animals and humans, represent emerging or re-emerging pathogens that pose significant public health threats throughout the world. It is therefore crucial to advance current surveillance mechanisms for these viruses through outlets such as phylogeography. Phylogeographic techniques may be applied to trace the origins and geographical distribution of these viruses using sequence and location data, which are often obtained from publicly available databases such as GenBank. Despite the abundance of zoonotic viral sequence data in GenBank records, phylogeographic analysis of these viruses is greatly limited by the lack of adequate geographic metadata. Although more detailed information may often be found in the related articles referenced in these records, manual extraction of this information presents a severe bottleneck. In this work, we propose an automated system for extracting this information using Natural Language Processing (NLP) methods. In order to validate the need for such a system, we first determine the percentage of GenBank records with ``insufficient'' geographic metadata for seven well-studied zoonotic viruses. We then evaluate four different named entity recognition (NER) systems which may help in the automatic extraction of information from related articles that can be used to improve the GenBank geographic metadata. This includes a novel dictionary-based location tagging system that we introduce in this paper.
