One of the most important features of health care is to be able to follow a progress over time and identify events in a temporal order. We describe initial steps in creating resources for automatic temporal reasoning of Swedish medical text. As a first step, we focus on the identification of temporal expressions, by exploiting existing resources and systems available for English. We adapt the HeidelTime system and manually evaluate its performance on a small subset of Swedish intensive care unit documents. On this subset, the adapted version of HeidelTime achieves a precision of 92\% and a recall of 66\%. We also extract the most frequent temporal expressions from a separate, larger subset, and note that most expressions concern parts of days or specific times. We intend to further develop resources for temporal reasoning of Swedish medical text by creating a gold standard also annotated with events and temporal links, in addition to temporal expressions and their normalised values.
