The present study aims to investigate the   application of prosodic speech features in a psychological intervention based on life-review. Several studies have shown that speech features can be used as indicators of depression severity, but these studies are mainly based on controlled speech recording tasks instead of natural conversations. The present exploratory study investigated speech features as indicators of depression in conversations of a therapeutic intervention. The changes in the prosodic speech features pitch, duration of pauses, and total duration of the participant's speaking time were studied over four sessions of a life-review inter-vention for three older participants. The ecological validity of the dynamics observed for prosodic speech features could not be established in the present study. The changes in speech features differed from what can be expected in an intervention that is effective in decreasing depression and were inconsistent with each other for each of the participants. We suggest future research to investigate changes within the intervention sessions, to relate the changes in feature values to the topical content of the speech, and to relate the speech features directly to depression scores.
