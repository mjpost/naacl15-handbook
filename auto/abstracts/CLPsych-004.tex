Violence risk assessment is an important and challenging task undertaken by mental health professionals and others, in both clinical and nonclinical settings. To date, computational linguistic techniques have not been used in the risk assessment process. However they could contribute to the current threat assessment process by allowing for early detection of elevated risk, identification of risk factors for violence, monitoring of violent intent, and determination of threat level. We analyzed a sample of communications to judges that were referred to security personnel for evaluation as constituting potential threats. We  categorized them along multiple dimensions including evidence of mental illness, presence and nature of any threat, and level of threat. While neither word count-based or topic models were able to effectively predict elevated risk, we found topics indicative of persecutory beliefs, paranoid ideation, and other symptoms of Axis I and Axis II disorders.
