The ubiquity of social media provides a rich opportunity to enhance the data available to mental health clinicians and researchers, enabling a better-informed and better-equipped mental health field. We present analysis of mental health phenomena in publicly available Twitter data, demonstrating how rigorous application of simple natural language processing methods can yield insight into specific disorders as well as mental health writ large, along with evidence that as-of-yet undiscovered linguistic signals relevant to mental health exist in social media. We present a novel method for gathering data for a range of mental illnesses quickly and cheaply, then focus on analysis of four in particular: post-traumatic stress disorder (PTSD), depression, bipolar disorder,  and seasonal affective disorder (SAD). We intend for these proof-of-concept results to inform the necessary ethical discussion regarding the balance between the utility of such data and the privacy of mental health related information.
