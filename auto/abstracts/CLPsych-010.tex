Mental illnesses such as depression and anxiety are highly prevalent, and therapy is increasingly being offered online.  This new setting is a departure from face-to-face therapy, and offers both a challenge and an opportunity -- it is not yet known what features or approaches are likely to lead to successful outcomes in such a different medium, but online text-based therapy provides large amounts of data for linguistic analysis.  We present an initial investigation into the application of computational linguistic techniques, such as topic and sentiment modelling, to online therapy for depression and anxiety. We find that important measures such as symptom severity can be predicted with comparable accuracy to face-to-face data, using general features such as discussion topic and sentiment; however, measures of patient progress are captured only by finer-grained lexical features, suggesting that aspects of style or dialogue structure may also be important.
