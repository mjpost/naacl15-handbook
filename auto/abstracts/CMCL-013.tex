The representations and processes yielding the limited length and telegraphic style of language production early on in acquisition have received little attention in acquisitional modeling. In this paper, we present a model, starting with minimal linguistic representations, that incrementally builds up an inventory of increasingly long and abstract grammatical representations (form+meaning pairings), in line with the usage-based conception of language acquisition. We explore its performance on a comprehension and a generation task, showing that, over time, the model better understands the processed utterances, generates longer utterances, and better expresses the situation these utterances intend to refer to.
