Infants spontaneously discover the relevant phonemes of their language without any direct supervision. This acquisition is puzzling because it seems to require the availability of high levels of linguistic structures (lexicon, semantics), that logically suppose the infants having a set of phonemes already. We show how this circularity can be broken by testing, in real-size language corpora, a scenario whereby infants would learn approximate representations at all levels, and then refine them in a mutual constraining way. We start with corpora of spontaneous speech that have been encoded in a varying number of detailed context-dependent allophones. We derive an approximate lexicon and a rudimentary semantic representation. Despite the fact that all these representations are poor approximations of the ground truth, they help reorganize the fine grained categories into phoneme-like categories with a high degree of accuracy.
