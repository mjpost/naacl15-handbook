This paper presents aspects of a computational model of the morphology of Plains Cree based on the technology of finite state transducers (FST). The paper focuses in particular on the modeling of nominal morphology. Plains Cree is a polysynthetic language whose nominal morphology relies on prefixes, suffixes and circumfixes. The model of Plains Cree morphology is capable of handling these complex affixation patterns and the morphophonological alternations that they engender. Plains Cree is an endangered Algonquian language spoken in numerous communities across Canada. The language has no agreed upon standard orthography, and exhibits widespread variation. We describe problems encountered and solutions found, while contextualizing the endeavor in the description, documentation and revitalization of First Nations Languages in Canada.
