In this paper, we investigate how topic dynamics during the course of an interaction correlate with the power differences between its participants. We perform this study on the US presidential debates and show that a candidate's power, modeled after their poll scores, affects how often he/she attempts to shift topics and whether he/she succeeds. We ensure the validity of topic shifts by confirming, through a simple but effective method, that the turns that shift topics provide substantive topical content to the interaction.
