Self-disclosure, the act of revealing oneself to others, is an important social behavior that contributes positively to intimacy and social support from others. It is a natural behavior, and social scientists have carried out numerous quantitative analyses of it through manual tagging and survey questionnaires. Recently, the flood of data from online social networks (OSN) offers a practical way to observe and analyze self-disclosure behavior at an unprecedented scale. The challenge with such analysis is that OSN data come with no annotations, and it would be impossible to manually annotate the data for a quantitative analysis of self-disclosure. As a solution, we propose a semi-supervised machine learning approach, using a variant of latent Dirichlet allocation for automatically classifying self-disclosure in a massive dataset of Twitter conversations. For measuring the accuracy of our model, we manually annotate a small subset of our dataset, and we show that our model shows significantly higher accuracy and F-measure than various other methods. With the results our model, we uncover a positive and significant relationship between self-disclosure and online conversation frequency over time.
