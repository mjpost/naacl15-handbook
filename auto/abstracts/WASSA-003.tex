In this research we focus on discriminating between emotive (emotionally loaded) and non-emotive sentences. We define the problem from a linguistic point of view assuming that emotive sentences stand out both lexically and grammatically. We verify this assumption experimentally by comparing two sets of such sentences in Japanese. The comparison is based on words, longer n-grams as well as more sophisticated patterns. In the classification we use a novel unsupervised learning algorithm based on the idea of language combinatorics. The method reached results comparable to the state of the art, while the fact that it is fully automatic makes it more efficient and language independent.
