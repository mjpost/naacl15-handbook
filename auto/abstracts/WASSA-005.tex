Irony is an important device in human communication, both in everyday spoken conversations as well as in written texts including books, websites, chats, reviews, and Twitter messages among others. Specific cases of irony and sarcasm have been studied in different contexts but to the best of our knowledge, only recently the first publicly available corpus including annotations about whether a text is ironic or not has been published by Filatova (2012). However, no baseline for classification of ironic or sarcastic reviews has been provided. With this paper, we aim at closing this gap. We formulate the problem as a supervised classification task and evaluate different classifiers, reaching an F1-measure of up to 74\% using logistic regression. We analyze the impact of a number of features which have been proposed in previous research as well as combinations of them.
