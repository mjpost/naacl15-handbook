In the past years, there has been an increasing amount of research done in the field of Senti-ment Analysis. This was motivated by the growth in the volume of user-generated online data, the information flood in Social Media and the applications Sentiment Analysis has to dif-ferent fields --- Marketing, Business Intelligence, e-Law Making, Decision Support Systems, etc. Although many methods have been proposed to deal with sentiment detection and classification in diverse types of texts and languages, many challenges still arise when passing these meth-ods from the research settings to real-life appli-cations. In this talk, we will describe the manner in which we employed machine translation together with human-annotated data to extend a sentiment analysis system to various languages. Addition-ally, we will describe how a joint multilingual model that detects and classifies sentiments ex-pressed in texts from Social Media has been de-veloped (at this point for Twitter and Facebook) and demo its use in a real-life application: a pro-ject aimed at detecting the citizens' attitude on Science and Technology.
