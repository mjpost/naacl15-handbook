Previous studies of the effect of word alignment on translation quality in SMT generally explore link level metrics only and mostly do not show any clear connections between alignment and SMT quality. In this paper, we specifically investigate the impact of word alignment on two pre-reordering tasks in translation, using a wider range of quality indicators than previously done. Experiments on German--English translation show that reordering may require alignment models different from those used by the core translation system. Sparse alignments with high precision on the link level, for translation units, and on the subset of crossing links, like intersected HMM models, are preferred. Unlike SMT performance the desired alignment characteristics are similar for small and large training data for the  pre-reordering tasks. Moreover, we confirm previous research showing that the fuzzy reordering score is a useful and cheap proxy for performance on SMT reordering tasks.
