Lexical substitution is an annotation task in which annotators provide one-word paraphrases (lexical substitutes) for individual target words in a sentence context. Lexical substitution yields a fine-grained characterization of word meaning that can be done by non-expert annotators. We discuss results of a recent lexical substitution annotation effort, where we found strong contextual modulation effects: Many substitutes were not synonyms, hyponyms or hypernyms of the targets, but were highly specific to the situation at hand. This data provides some food for thought for frame-semantic analysis.
