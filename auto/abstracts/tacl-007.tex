Stress has long been established as a major cue in word segmentation for English infants. We show that enabling a current state-of-the-artBayesian word segmentation model to take advantage of stress cuesnoticeably improves its performance. We find that the improvementsrange from 10 to 4\%, depending on both the use of phonotactic cuesand, to a lesser extent, the amount of evidence available to the learner.We also find that in particular early on, stress cues are much moreuseful for our model than phonotactic cues by themselves, consistent with the finding that children do seem to use stress cues before they use phonotactic cues. Finally, we study how the model's knowledge about stress patterns evolves over time. We not only find that our model correctly acquires the most frequent patterns relatively quickly but also that the Unique Stress Constraint that is at the heart of a previously proposed model does not need to be built in but can be acquired jointly with word segmentation.
