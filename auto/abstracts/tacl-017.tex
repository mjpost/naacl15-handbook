Greedy transition-based parsers are very fast but tend to suffer from error propagation. This problem is aggravated by the fact that they are normally trained using oracles that are deterministic and incomplete in the sense that they assume a unique canonical path through the transition system and are only valid as long as the parser does not stray from this path. In this paper, we give a general characterization of oracles that are nondeterministic and complete, present a method for deriving such oracles for transition systems that satisfy a property we call arc decomposition, and instantiate this method for three well-known transition systems from the literature. We say that these oracles are dynamic, because they allow us to dynamically explore alternative and nonoptimal paths during training --- in contrast to oracles that statically assume a unique optimal path. Experimental evaluation on a wide range of data sets clearly shows that using dynamic oracles to train greedy parsers gives substantial improvements in accuracy. Moreover, this improvement comes at no cost in terms of efficiency, unlike other techniques like beam search.
