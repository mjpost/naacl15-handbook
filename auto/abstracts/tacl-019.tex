In this paper, we study the problem of manually correcting automatic annotations of natural language in as efficient a manner as possible. We introduce a method for automatically segmenting a corpus into chunks such that many uncertain labels are grouped into the same chunk, while human supervision can be omitted altogether for other segments. A tradeoff must be found for segment sizes. Choosing short segments allows us to reduce the number of highly confident labels that are supervised by the annotator, which is useful because these labels are often already correct and supervising correct labels is a waste of effort. In contrast, long segments reduce the cognitive effort due to context switches. Our method helps find the segmentation that optimizes supervision efficiency by defining user models to predict the cost and utility of supervising each segment and solving a constrained optimization problem balancing these contradictory objectives. A user study demonstrates substantial gains over pre-segmented, confidence-ordered baselines on two natural language processing tasks: speech transcription and word segmentation.
