%
% File acl2014.tex%
% Contact: koller@ling.uni-potsdam.de, yusuke@nii.ac.jp
%%
%% Based on the style files for ACL-2013, which were, in turn,
%% Based on the style files for ACL-2012, which were, in turn,
%% based on the style files for ACL-2011, which were, in turn, 
%% based on the style files for ACL-2010, which were, in turn, 
%% based on the style files for ACL-IJCNLP-2009, which were, in turn,
%% based on the style files for EACL-2009 and IJCNLP-2008...

%% Based on the style files for EACL 2006 by 
%%e.agirre@ehu.es or Sergi.Balari@uab.es
%% and that of ACL 08 by Joakim Nivre and Noah Smith

\documentclass[11pt]{article}
\usepackage{acl2014}
\usepackage{times}
\usepackage{latexsym}
\usepackage{url}
\usepackage{latexsym}

%\setlength\titlebox{5cm}

% You can expand the titlebox if you need extra space
% to show all the authors. Please do not make the titlebox
% smaller than 5cm (the original size); we will check this
% in the camera-ready version and ask you to change it back.


\title{Improving Twitter Sentiment Analysis with Topic-Based Mixture Modeling and Semi-Supervised Training}

\author{Bing Xiang *\\
  IBM Watson \\
  1101 Kitchawan Rd \\
  Yorktown Heights, NY 10598, USA \\
  {\tt bingxia@us.ibm.com} \\\And
  Liang Zhou \\
  Thomson Reuters \\
  3 Times Square \\
  New York, NY 10036, USA \\
  {\tt l.zhou@thomsonreuters.com} \\}

\date{}

\begin{document}
\maketitle
\begin{abstract}
  In this paper, we present multiple approaches to improve sentiment analysis on Twitter data.
  We first establish a state-of-the-art baseline with a rich feature set. Then we build a topic-based sentiment
  mixture model with topic-specific data in a semi-supervised training framework. The topic information
  is generated through topic modeling based on an efficient implementation of Latent Dirichlet Allocation (LDA).
  The proposed sentiment model outperforms the top system in the task of
  \textit{Sentiment Analysis in Twitter} in SemEval-2013 in terms of averaged F scores.
\let\thefootnote\relax\footnote{* This work was done when the author was with Thomson Reuters.}
\end{abstract}

\section{Introduction}

Social media, such as Twitter and Facebook, has attracted significant attention in recent years.
The vast amount of data available online provides a unique opportunity to the people working on natural 
language processing (NLP) and related fields. Sentiment analysis is one of the areas that has large potential
in real-world applications. For example, monitoring the trend of sentiment 
for a specific company or product mentioned in social media can be useful in stock prediction and product marketing.

In this paper, we focus on sentiment analysis of Twitter data (tweets). It is one of the 
challenging tasks in NLP given the length limit on each tweet (up to 140 characters) and also
the informal conversation. Many approaches have been proposed previously to improve sentiment
analysis on Twitter data. For example, Nakov et al. (2013) provide an overview on the systems
submitted to one of the SemEval-2013 tasks, \textit{Sentiment Analysis in Twitter}. A variety of
features have been utilized for sentiment classification on tweets. They include lexical features (e.g. word lexicon),
syntactic features (e.g. Part-of-Speech), Twitter-specific features (e.g. emoticons), etc.
However, all of these features only capture local information in the data and do not take into
account of the global higher-level information, such as topic information. 

Two example tweets are given below, with the word ``\textit{offensive}" appearing in both of them.

\begin{itemize}
\item \textit{Im gonna post something that might be {\bf offensive} to people in Singapore.}
\item \textit{\#FSU {\bf offensive} coordinator Randy Sanders coached for Tennessee in 1st \#BCS title game.}
\end{itemize}
Generally ``\textit{offensive}" is used as a negative word (as in the first tweet), but it bears no sentiment 
in the second tweet when people are talking about a football game.
Even though some local contextual features could be helpful to distinguish the two cases above,
they still may not be enough to get the sentiment on the whole message correct. 
Also, the local features often suffer from the sparsity problem.
This motivates us to explore topic information explicitly in the task of sentiment analysis on Twitter data.

There exists some work on applying topic information in sentiment analysis,
such as \cite{Mei:07}, \cite{Branavan:08}, \cite{Jo:11} and \cite{He:12}.
All these work are significantly different from what we propose in this work. Also they are 
conducted in a domain other than Twitter. Most recently, Si et al. (2013) propose a continuous Dirichlet Process
Mixture model for Twitter sentiment, for the purpose of stock prediction. Unfortunately there is no evaluation on the
accuracy of sentiment classification alone in that work. Furthermore, no standard
training or test corpus is used, which makes comparison with other approaches difficult.

Our work is organized in the following way:

\begin{itemize}
\item We first propose a universal sentiment model that utilizes various features and resources. The universal
model outperforms the top system submitted to the SemEval-2013 task \cite{Mohammad:13}, 
which was trained and tested
on the same data. The universal model serves as a strong baseline and also provides an option for smoothing later.
\item We introduce a topic-based mixture model for Twitter sentiment. The model is integrated in the framework of
semi-supervised training that takes advantage of large amount of un-annotated Twitter data. Such a mixture model
results in further improvement on the sentiment classification accuracy.
\item We propose a smoothing technique through interpolation between universal model and topic-based mixture model.
\item We also compare different approaches for topic modeling, such as cross-domain topic identification by
utilizing data from newswire domain.
\end{itemize}

\section{Universal Sentiment Classifier}
In this section we present a universal topic-independent sentiment classifier to establish a state-of-the-art baseline.
The sentiment labels are either positive, neutral or negative.

\subsection{SVM Classifier}
Support Vector Machine (SVM) is an effective classifier that can achieve good performance in high-dimensional
feature space. An SVM model represents the examples as points in space, mapped so that the examples of the 
different categories are separated by a clear margin as wide as possible. In this work an SVM classifier is trained
with LibSVM \cite{LibSVM:11}, a widely used toolkit. The linear kernel is found to achieve higher accuracy than other
kernels in our initial experiments. The option of probability estimation in LibSVM is turned on so that it can produce the
probability of sentiment class $c$ given tweet $x$ at the classification time, i.e. $P(c|x)$.

\subsection{Features}
The training and testing data are run through tweet-specific tokenization, similar to that used in the CMU Twitter NLP
tool \cite{Gimpel:11}. It is shown in Section 5 that such customized tokenization is helpful.
Here are the features that we use for classification:

\begin{itemize}
\item Word N-grams: if certain N-gram (unigram, bigram, trigram or 4-gram) appears in the tweet, the corresponding
feature is set to 1, otherwise 0. These features are collected from training data, with a count cutoff to
avoid overtraining.
\item Manual lexicons: it has been shown in other work \cite{SemEval:13} that lexicons with positive and negative words
are important to sentiment classification. In this work, we adopt the lexicon from Bing Liu \cite{Liu:04} which includes
about 2000 positive words and 4700 negative words. We also experimented with the popular MPQA \cite{Wilson:05} lexicon but
found no extra improvement on accuracies. A short list of Twitter-specific positive/negative words are also added to enhance
the lexicons. We generate two features based on the lexicons: total number of positive words or
negative words found in each tweet.
\item Emoticons: it is known that people use emoticons in social media data to express their
emotions. A set of popular emoticons are collected from the Twitter data we have. Two features are created to
represent the presence or absence of any positive/negative emoticons.
\item Last sentiment word: a ``sentiment word" is any word in the positive/negative lexicons mentioned
above. If the last sentiment word found in the tweet is positive (or negative), this feature is set to 1 (or -1). If
none of the words in the tweet is sentiment word, it is set to 0 by default.
\item PMI unigram lexicons: in \cite{Mohammad:13} two lexicons were automatically generated based on pointwise mutual 
information (PMI). One is \textit{NRC Hashtag Sentiment Lexicon} with 54K unigrams, and the other is
\textit{Sentiment140 Lexicon} with 62K unigrams. Each word in the lexicon has an associated sentiment score.
We compute 7 features based on each of the two lexicons: (1) sum of sentiment score;
(2) total number of positive words (with score $s>1$); (3) total number of negative words ($s<-1$);
(4) maximal positive score; (5) minimal negative score; (6) score of the last positive words; 
(7) score of the last negative words.
Note that for the second and third features, we ignore those
with sentiment scores between -1 and 1, since we found that inclusion of those weak subjective words results in
unstable performance.
\item PMI bigram lexicon: there are also 316K bigrams in the \textit{NRC Hashtag Sentiment Lexicon}. For bigrams,
we did not find the sentiment scores useful. Instead, we only compute two features based on counts only: 
total number of positive bigrams; total number of negative bigrams. 
\item Punctuations: if there exists exclamation mark or question mark in the tweet, the feature is set to 1, otherwise
set to 0.
\item Hashtag count: the number of hashtags in each tweet.
\item Negation: we collect a list of negation words, including some informal words frequently observed in
online conversations, such as ``\textit{dunno}" (``don't know"), ``\textit{nvr}" (``never"), etc. For any sentiment words
within a window following a negation word and not after punctuations `.', `,', `;', `?', or `!', we reverse 
its sentiment from positive to negative, or vice versa, before computing the lexicon-based features 
mentioned earlier. The window size was set to 4 in this work.
\item Elongated words: the number of words in the tweet that have letters repeated by at least 3 times in a row, e.g. 
the word ``\textit{gooood}".
\end{itemize}

\section{Topic-Based Sentiment Mixture}

\subsection{Topic Modeling}
Latent Dirichlet Allocation (LDA) \cite{Blei:03} is one of the widely adopted generative models for topic modeling. 
The fundamental idea is that a document is a mixture of topics. For each document there is a multinomial
distribution over topics, and a Dirichlet prior $Dir({\bf \alpha})$ is introduced on such distribution. 
For each topic, there is
another multinomial distribution over words.  One of the popular algorithms for LDA model parameter 
estimation and inference is Gibbs sampling \cite{Griffiths:04}, a form of Markov Chain Monte Carlo.
We adopt the efficient implementation of Gibbs sampling as proposed in \cite{Yao:09} in this work.

Each tweet is regarded as one document. We conduct pre-processing by removing stop words and some of the frequent words 
found in Twitter data. Suppose that there are $T$ topics in total in the training data, 
i.e. $t_1$, $t_2$, ..., $t_T$. The posterior probability of each topic given tweet $x_i$ is computed as in Eq.~\ref{eq1}: 

\begin{equation}
\label{eq1}
P_t(t_j|x_i) = \frac{C_{ij}+\alpha_j}{\sum^T_{k=1}C_{ik}+T\alpha_j}
\end{equation}
where $C_{ij}$ is the number of times that topic $t_j$ is assigned to some word in tweet $x_i$, usually
averaged over multiple iterations of Gibbs sampling. $\alpha_j$ is the
$j$-th dimension of the hyperparameter of Dirichlet distribution that can be optimized during model estimation.

\subsection{Sentiment Mixture Model}
Once we identify the topics for tweets in the training data, we can split the data into multiple subsets
based on topic distributions. For each subset, a separate sentiment model can be trained. There are many
ways of splitting the data. For example, K-means clustering can be conducted based on the similarity between
the topic distribution vectors or their transformed versions. In this work, we assign tweet $x_i$ to cluster $j$
if $P_t(t_j|x_i) > \tau$ or $P_t(t_j|x_i) = \max_k P_t(t_k|x_i)$. Note that this is a soft clustering, with some
tweets possibily assigned to multiple topic-specific clusters. Similar to the universal model, we train $T$
topic-specific sentiment models with LibSVM. 

During classification on test tweets, we run topic inference
and sentiment classification with multiple sentiment models. They jointly determine the final probability
of sentiment class $c$ given tweet $x_i$ as the following in a sentiment mixture model:

\vspace{-5 mm}
\begin{equation}
P(c|x_i) = \sum_{j=1}^TP_m(c|t_j,x_i)P_t(t_j|x_i)
\end{equation}
where $P_m(c|t_j,x_i)$ is the probability of sentiment $c$ from topic-specific sentiment model trained on topic $t_j$.

\subsection{Smoothing}
Additionally, we also experiment with a smoothing technique through linear interpolation between the universal
sentiment model and topic-based sentiment mixture model.

\vspace{-5 mm}
\begin{eqnarray}
P(c|x_i) &=& \theta \times P_U(c|x_i)+(1-\theta) \nonumber \\ 
         &\times& \sum_{j=1}^TP_m(c|t_j,x_i)P_t(t_j|x_i)
\end{eqnarray}
where $\theta$ is the interpolation parameter and $P_U(c|x_i)$ is the probability of sentiment $c$ given tweet $x_i$
from the universal sentiment model.

\section{Semi-supervised Training}
In this section we propose an integrated framework of semi-supervised training that contains both topic modeling
and sentiment classification. The idea of semi-supervised training is to take advantage of large amount low-cost
un-annotated data (tweets in this case) to further improve the accuracy of sentiment classification. The algorithm
is as follows:

\begin{enumerate}
\item Set training corpus $D$ for sentiment classification to be the annotated training data $D_a$;
\item Train a sentiment model with current training corpus $D$; 
\item Run sentiment classification on the un-annotated data $D_u$ with the current sentiment model and 
generate probabilities of sentiment classes for each tweet, $P(c|x_i)$;
\item Perform data selection. For those tweets with $P(c|x_i) > p$, add them to current training corpus $D$. The rest
is used to replace the un-annotated corpus $D_u$;
\item Train a topic model on $D$, and store the topic inference model and topic distributions of each tweet; 
\item Cluster data in $D$ based on the topic distributions from Step 5 and train a separate sentiment model for
each cluster. Replace current sentiment model with the new sentiment mixture model;
\item Repeat from Step 3 until finishing a pre-determined number of iterations or no more data is added to $D$ in Step 4. 
\end{enumerate}

\section{Experimental Results}

\subsection{Data and Evaluation}
We conduct experiments on the data from the task B of \textit{Sentiment Analysis in Twitter}
in SemEval-2013. The distribution of positive, neutral and negative data is shown in Table~\ref{data}.
The development set is used to tune parameters and features. The test set is for the blind evaluation.

\begin{table}[h]
\begin{center}
\begin{tabular}{|l|r|r|r|r|}
\hline \bf Set & \bf Pos & \bf Neu & \bf Neg & \bf Total \\ \hline\hline
Training & 3640 & 4586 & 1458 & 9684 \\ \hline
Dev & 575 & 739 & 340 & 1654 \\ \hline
Test & 1572 & 1640 & 601 & 3813 \\ \hline
\end{tabular}
\end{center}
\caption{\label{data} Data from SemEval-2013. Pos: positive; Neu: neutral; Neg: negative.}
\end{table}

For semi-supervised training experiments, we
explored two sets of additional data. The first one contains 2M tweets randomly sampled from the collection in January 
and February 2014. The other
contains 74K news documents with 50M words collected during the first half year of 2013 from online newswire.

For evaluation, we use macro averaged F score as in \cite{SemEval:13}, i.e. average of the F scores computed on
positive and negative classes only. Note that this does not make the task a binary classification problem.
Any errors related to neutral class (false positives or false negatives) will negatively impact the F scores.

\subsection{Universal Model}
In Table~\ref{universal}, we show the incremental improvement in adding various features described in Section 2,
measured on the test set. 
In addition to the features, we also find SVM weighting on the training samples is helpful. Due to the
skewness in class distribution in the training set, it is observed during error analysis on the development
set that subjective (positive/negative) tweets are more likely to be classified as neutral tweets. The
weights for positive, neutral and negative samples are set to be (1, 0.4, 1) based on the
results on the development set. As shown in Table~\ref{universal}, weighting adds a 2\% improvement.
With all features combined, the universal sentiment model achieves 69.7 on average F score.
The F score from the best system in SemEval-2013 \cite{Mohammad:13} is also listed in the last row of
Table~\ref{universal} for a comparison.

\begin{table}[h]
\begin{center}
\begin{tabular}{|l|c|}
\hline \bf Model & \bf Avg. F score \\ \hline\hline
Baseline with word N-grams & 55.0 \\ \hline
+ tweet tokenization & 56.1 \\ \hline
+ manual lexicon features & 62.4 \\ \hline
+ emoticons & 62.8 \\ \hline
+ last sentiment word & 63.7 \\ \hline
+ PMI unigram lexicons & 64.5 \\ \hline
+ hashtag counts & 65.0 \\ \hline
+ SVM weighting & 67.0 \\ \hline
+ PMI bigram lexicons & 68.2 \\ \hline
+ negations & 69.0 \\ \hline
+ elongated words & 69.7 \\ \hline\hline
Mohammad et al., 2013 & 69.0 \\ \hline
\end{tabular}
\end{center}
\caption{\label{universal} Results on the test set with universal sentiment model.}
\end{table}

\subsection{Topic-Based Mixture Model}

For the topic-based mixture model and semi-supervised training, based on the experiments on the development set,
we set the parameter $\tau$ used in soft clustering to 0.4, the data selection parameter $p$ to 0.96, and the 
interpolation parameter for smoothing $\theta$ to 0.3. We found no more noticeable benefits after two iterations
of semi-supervised training. The number of topics is set to 100.

The results on the test set are shown Table~\ref{mixture}, with the topic information
inferred from either Twitter data (second column) or newswire data (third column).
The first row shows the performance of the universal sentiment model as a baseline.
The second row shows the results from re-training the universal model by simply adding tweets
selected from two iterations of semi-supervised training (about 100K). It serves as another
baseline with more training data, for a fair comparison with the topic-based mixture modeling that uses
the same amount of training data. 

We also conduct an experiment by only considering the most likely topic for each tweet when computing the sentiment
probabilities. The results show that the topic-based mixture model outperforms both the baseline and the one
that considers the top topics only. Smoothing with the universal model adds further improvement in addition to
the un-smoothed mixture model. With the topic information inferred from Twitter data, the F score is 2 points 
higher than the baseline without semi-supervised training and 1.4 higher than the baseline with semi-supervised data. 

\begin{table}[t]
\begin{center}
\begin{tabular}{|l|c|c|} \hline 
\bf Model  & \bf Tweet-topic & \bf News-topic \\ \hline\hline
Baseline          & 69.7 & 69.7 \\ \hline
+ semi-supervised & 70.3 & 70.2 \\ \hline\hline
top topic only    & 70.6 & 70.4 \\ \hline
mixture           & 71.2 & 70.8 \\ \hline
+ smoothing       & 71.7 & 71.1 \\ \hline
\end{tabular}
\end{center}
\caption{\label{mixture} Results of topic-based sentiment mixture model on SemEval test set.}
\end{table}

As shown in the third column in Table~\ref{mixture}, surprisingly, the model with topic information inferred from the newswire 
data works well on the Twitter domain. A 1.4 points of improvement can be obtained compared to the baseline.
This provides an opportunity for cross-domain topic identification when data from certain domain is more difficult 
to obtain than others.

In Table~\ref{topics}, we provide some examples from the topics identified in tweets as well as the newswire
data. The most frequent words in each topic are listed in the table. We can clearly see that the topics are about
phones, sports, sales and politics, respectively.

\begin{table}[h]
\begin{center}
\begin{tabular}{|c|c|c|c|c|}
\hline
\bf Tweet-1 & \bf Tweet-2 & \bf News-1 & \bf News-2 \\ \hline\hline
phone     &  game  &  sales  &  party  \\
call      &  great &  stores &  government \\
answer    &  play  &  online &  election \\
question  &  team  &  retail &  minister \\
service   &  win   &  store  &  political \\ 
text      &  tonight & retailer &  prime \\
texting   &  super &  business &  state \\
\hline
\end{tabular}
\end{center}
\caption{\label{topics} The most frequent words in example topics from tweets and newswire data.}
\end{table}

\section{Conclusions}
In this paper, we presented multiple approaches for advanced Twitter sentiment
analysis. We established a state-of-the-art baseline that utilizes a variety of features, 
and built a topic-based sentiment mixture model with topic-specific Twitter data, all
integrated in a semi-supervised training framework. 
The proposed model outperforms the top system in SemEval-2013. 
Further research is needed to continue to improve the accuracy in this difficult domain.

% include your own bib file like this:
%\bibliographystyle{acl}
%\bibliography{acl2014}

\begin{thebibliography}{}
\bibitem[\protect\citename{Blei \bgroup et al.\egroup }2003]{Blei:03}
David Blei, Andrew Y. Ng, and Michael I. Jordan.
\newblock 2003.
\newblock Latent Dirichlet allocation.
\newblock {\em In Journal of Machine Learning Research.}
3(2003), 993--1022.

\bibitem[\protect\citename{Branavan \bgroup et al.\egroup }2008]{Branavan:08}
S. R. K. Branavan, Harr Chen, Jacob Eisenstein, and Regina Barzilay.
\newblock 2008.
\newblock Learning document-level semantic properties from free-text annotations.
\newblock {\em In Proceedings of the Annual Meeting of the Association for Computational Linguistics (ACL-2008).}

\bibitem[\protect\citename{Chang and Lin}2011]{LibSVM:11}
Chih-Chung Chang and Chih-Jen Lin.
\newblock 2011.
\newblock LIBSVM: A library for support vector machines.
\newblock {\em In ACM Transactions on Intelligent Systems and Technology.}

\bibitem[\protect\citename{Gimpel \bgroup et al.\egroup }2011]{Gimpel:11}
Kevin Gimpel, Nathan Schneider, Brendan O'Connor, Dipanjan Das, Daniel Mills, Jacob Eisenstein, Michael Heilman, Dani Yogatama, Jeffrey Flanigan, and Noah A. Smith.
\newblock 2011.
\newblock Part-of-speech tagging for Twitter: annotation, features, and experiments.
\newblock {\em In Proceedings of the Annual Meeting of the Association for Computational Linguistics.}

\bibitem[\protect\citename{Griffiths and Steyvers}2004]{Griffiths:04}
Thomas L. Griffiths and Mark Steyvers.
\newblock 2004.
\newblock Finding scientific topics.
\newblock {\em In Proceedings of the National Academy of Science.}
101, 5228--5235.

\bibitem[\protect\citename{He \bgroup et al.\egroup }2012]{He:12}
Yulan He, Chenghua Lin, Wei Gao, and Kam-Fai Wong.
\newblock 2012.
\newblock Tracking sentiment and topic dynamics from social media.
\newblock {\em In Proceedings of the 6th International AAAI Conference on Weblogs and Social Media (ICWSM-2012).}

\bibitem[\protect\citename{Hu and Liu}2004]{Liu:04}
Mingqing Hu and Bing Liu.
\newblock 2004.
\newblock Mining and summarizing customer reviews.
\newblock {\em In Proceedings of the Tenth ACM SIGKDD International Conference on Knowledge Discovery and Data Mining.}

\bibitem[\protect\citename{Jo and Oh}2011]{Jo:11}
Yohan Jo and Alice Oh.
\newblock 2011.
\newblock Aspect and sentiment unification model for online review analysis.
\newblock {\em In Proceedings of ACM Conference in Web Search and Data Mining (WSDM-2011).}

\bibitem[\protect\citename{Mei \bgroup et al.\egroup }2007]{Mei:07}
Qiaozhu Mei, Xu Ling, Matthew Wondra, Hang Su, and ChengXiang Zhai.
\newblock 2007.
\newblock Topic sentiment mixture: modeling facets and opinions in weblogs.
\newblock {\em In Proceedings of International Conference on World Wide Web (WWW-2007).}

\bibitem[\protect\citename{Mohammad \bgroup et al.\egroup }2013]{Mohammad:13}
Saif M. Mohammad, Svetlana Kiritchenko, and Xiaodan Zhu.
\newblock 2013.
\newblock NRC-Canada: Building the state-of-the-art in sentiment analysis of tweets.
\newblock {\em In Proceedings of the Seventh International Workshop on Semantic Evaluation (SemEval 2013).}
312-–320, Atlanta, Georgia, June 14-15, 2013.

\bibitem[\protect\citename{Nakov \bgroup et al.\egroup }2013]{SemEval:13}
Preslav Nakov, Zornitsa Kozareva, Alan Ritter, Sara Rosenthal, Veselin Stoyanov, and Theresa Wilson.
\newblock 2013.
\newblock SemEval-2013 Task 2: Sentiment Analysis in Twitter.
\newblock {\em In Proceedings of the Seventh International Workshop on Semantic Evaluation (SemEval 2013).}
312-–320, Atlanta, Georgia, June 14-15, 2013.

\bibitem[\protect\citename{Si \bgroup et al.\egroup }2013]{Si:13}
Jianfeng Si, Arjun Mukherjee, Bing Liu, Qing Li, Huayi Li, and Xiaotie Deng.
\newblock 2013.
\newblock Exploiting topic based Twitter sentiment for stock prediction.
\newblock {\em In Proceedings of the 51st Annual Meeting of the Association for Computational Linguistics.}
24-29, Sofia, Bulgaria, August 4-9,2013.

\bibitem[\protect\citename{Wilson \bgroup et al.\egroup }2005]{Wilson:05}
Theresa Wilson, Janyce Wiebe, and Paul Hoffmann.
\newblock 2005. 
\newblock Recognizing contextual polarity in phrase-level sentiment analysis.
\newblock {\em In Proceedings of the Conference on Human Language Technology and Empirical Methods in Natural Language Processing, HLT ’05.}

\bibitem[\protect\citename{Yao \bgroup et al.\egroup }2009]{Yao:09}
Limin Yao, David Mimno, and Andrew McCallum.
\newblock 2009.
\newblock Efficient methods for topic model inference on streaming document collections.
\newblock {\em KDD'09}.

\end{thebibliography}
\end{document}
