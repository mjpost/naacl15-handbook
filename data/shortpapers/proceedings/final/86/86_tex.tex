%
% File acl2014.tex
%
% Contact: koller@ling.uni-potsdam.de, yusuke@nii.ac.jp
%%
%% Based on the style files for ACL-2013, which were, in turn,
%% Based on the style files for ACL-2012, which were, in turn,
%% based on the style files for ACL-2011, which were, in turn, 
%% based on the style files for ACL-2010, which were, in turn, 
%% based on the style files for ACL-IJCNLP-2009, which were, in turn,
%% based on the style files for EACL-2009 and IJCNLP-2008...

%% Based on the style files for EACL 2006 by 
%%e.agirre@ehu.es or Sergi.Balari@uab.es
%% and that of ACL 08 by Joakim Nivre and Noah Smith

\documentclass[11pt]{article}
\usepackage{acl2014}
\usepackage{times}
\usepackage{url}
\usepackage{latexsym}
\usepackage{amsmath,amssymb,threeparttable}
\usepackage{pst-plot}
%\usepackage[dvips]{graphics}
\usepackage[dvipdfm]{graphicx}
\usepackage{bmpsize}

%\setlength\titlebox{6.5cm}    % You can expand the title box if you
% really have to

\title{Empirical Study of Unsupervised Chinese Word Segmentation Methods for SMT on Large-scale Corpora}

\author{Xiaolin Wang \hspace{.7cm} Masao Utiyama \hspace{.7cm} Andrew Finch \hspace{.7cm} Eiichiro Sumita \\
National Institute of Information and Communications Technology \\
{\tt \{xiaolin.wang,mutiyama,andrew.finch,eiichiro.sumita\}@nict.go.jp } \\}

%xiaolin wang, Masao Utiyama, Andrew Finch and Eiichiro Sumita 
%\author{First Author \\
%  Affiliation / Address line 1 \\
%  Affiliation / Address line 2 \\
%  Affiliation / Address line 3 \\
%  {\tt email@domain} \\\And
%  Second Author \\
%  Affiliation / Address line 1 \\
%  Affiliation / Address line 2 \\
%  Affiliation / Address line 3 \\
%  {\tt email@domain} \\}


\date{}

\begin{document}

\maketitle
\begin{abstract}

Unsupervised word segmentation (UWS) can provide domain-adaptive segmentation for statistical machine translation (SMT) without annotated data, and bilingual UWS can even optimize segmentation for alignment. Monolingual UWS approaches of explicitly modeling the probabilities of words through Dirichlet process (DP) models or Pitman-Yor process (PYP) models have achieved high accuracy, but their bilingual counterparts have only been carried out on small corpora such as basic travel expression corpus (BTEC) due to the computational complexity.  This paper proposes an efficient unified PYP-based monolingual and bilingual UWS method. Experimental results show that the proposed method is comparable to supervised segmenters on the in-domain NIST OpenMT corpus, and yields a 0.96 BLEU relative increase on NTCIR PatentMT corpus which is out-of-domain.
\end{abstract}



\section{Introduction}

Many languages, especially Asian languages such as Chinese, Japanese and
Myanmar, have no explicit word boundaries, thus word segmentation (WS),
that is, segmenting the continuous texts of these languages into
isolated words, is a prerequisite for many natural language processing
applications including SMT.

Though supervised-learning approaches which involve training segmenters on manually segmented corpora are widely used~\cite{chang2008optimizing}, yet the criteria for manually annotating words are arbitrary, and the available annotated corpora are limited in both quantity and genre variety. For example, in machine translation, there
are various parallel corpora such as BTEC for tourism-related
dialogues~\cite{paul2008overview} and PatentMT in the patent 
domain~\cite{goto2011overview}\footnote{\url{http://ntcir.nii.ac.jp/PatentMT}}, 
but researchers working on Chinese-related tasks often use the Stanford Chinese
segmenter~\cite{tseng2005conditional} which is
trained on a small amount of annotated news text.

In contrast, UWS, spurred by the findings that infants are able to use
statistical cues to determine word
boundaries~\cite{saffran1996statistical}, relies on statistical criteria
instead of manually crafted standards.  UWS learns from unsegmented raw
text, which are available in large quantities, and thus it has the potential
to provide more accurate and adaptive segmentation than supervised
approaches with less development effort being required.

The approaches of explicitly modeling the probability of
words\cite{brent1999efficient,venkataraman2001statistical,goldwater2006contextual,goldwater2009bayesian,mochihashi2009bayesian}
significantly outperformed a heuristic approach~\cite{zhao2008empirical}
on the monolingual Chinese SIGHAN-MSR corpus~\cite{emerson2005second},
which inspired the work of this paper.

However, bilingual approaches that model word probabilities suffer from
computational complexity. Xu et al.~\shortcite{xu2008bayesian} proposed
a bilingual method by adding alignment into the generative model, but
was only able to test it on small-scale BTEC data. Nguyen et al.~\shortcite{nguyen2010nonparametric}  
used the local best alignment to increase the speed of the Gibbs sampling in training but the impact
on accuracy was not explored.

This paper is dedicated to bilingual UWS  on large-scale corpora to
support SMT. To this end, we model bilingual UWS under a similar framework
with monolingual UWS in order to improve efficiency,
and replace Gibbs sampling with expectation maximization (EM) in training.  

We aware that variational bayes (VB) may be used for speeding up the
training of DP-based or PYP-based bilingual UWS. However, VB requires
formulating  the $m$ expectations of  $(m-1)$-dimensional marginal
distributions, where $m$ is the number of hidden variables. For UWS, the
hidden variables are indicators that identify substrings of sentences in
the corpus as words. These variables are large in number and it is not
clear how to apply VB to UWS, and as far the authors aware there is
no previous work related to the application of VB to monolingual UWS.
Therefore, we have not explored VB methods in this paper, but we do show
that our method is superior to the existing methods.

The contributions of this paper include,
\begin{itemize}
\item state-of-the-art accuracy in monolingual UWS;
\item the first bilingual UWS method practical for large corpora;
\item improvement of BLEU scores compared to supervised Stanford Chinese word segmenter.
\end{itemize}

\section{Methods}

This section describes our unified monolingual and bilingual UWS scheme.
Table~\ref{tab:notation} lists the main notation. The set $\mathcal{F}$ is
chosen to represent an unsegmented foreign language sentence (a
sequence of characters), because an unsegmented sentence can be seen as the
set of all possible segmentations of the sentence denoted $F$, i.e. $F
\in \mathcal{F}$.

\begin{table}[h]
{
\begin{center}
\begin{tabular}{|l|l|}
\hline
 \bf Notation & \bf Meaning  \\ \hline
$\mathcal{F}$ & an unsegmented foreign sentence \\
$\mathcal{F}_k^{k'}$ &unsegmented substring of the un-\\
                     &derlying string of $\mathcal{F}$ from $k$ to $k'$ \\
$F$ &a segmented foreign sentence \\
$f_j$ & the $j$-th foreign word \\
$\mathcal{M}$ &monolingual segmentation model \\
$P_{\small \mathcal{M}}(x)$ & probability of $x$ being a word ac-\\
                            &cording to $M$ \\
$E$ & a tokenized English sentence \\
$e_i$ & the $i$-th English word \\
($\mathcal{F}$,$E$) &a bilingual sentence pair \\
$\mathcal{B}$ & bilingual segmentation model \\
$P_{\small \mathcal{B}}(x|e_i)$ & probability of $x$ being a word ac-\\
                              &cording to $B$ given $e_i$ \\
\hline
\end{tabular}
\end{center}
}
\caption{\label{tab:notation} Main Notation. }
\end{table}

Monolingual and bilingual WS can be formulated as follows, respectively,
\begin{align}
&\hat{F}(\mathcal{F})  = \mathop{argmax}_{F  \in \mathcal{F} } P(F|\mathcal{F},\mathcal{M}), \label{eq:segment:unigram} \\
&\hat{F}(\mathcal{F},E)  =\mathop{argmax}_{F \in \mathcal{F} } \sum_a P(F,a|\mathcal{F},E,\mathcal{B}),  \label{eq:segment:biling}
\end{align}

where $a$ is an alignment between $F$ and $E$. The English sentence $E$
is used in the generation of a segmented sentence $F$.


UWS learns models by maximizing the likelihood of the unsegmented corpus, formulated as,
\begin{align}
&\hat{\mathcal{M}}=\mathop{\mathrm{argmax}}_\mathcal{M} \prod_{\mathcal{F} \in \mathbb{F} } \Big( \sum_{F \in \mathcal{F} } P(F|\mathcal{M}) \Big), \label{eq:model:monoling}  \\
&\hat{\mathcal{B}}=\mathop{\mathrm{argmax}}_\mathcal{B} \prod_{(\mathcal{F},E) \in \mathbb{B} } \Big( \sum_{F \in \mathcal{F} } \sum_a P(F,a|\mathcal{F},E,\mathcal{B})\Big). \label{eq:model:biling}  
\end{align}

Our method of learning $\mathcal{M}$ and $\mathcal{B}$ proceeds in a similar manner to the
EM algorithm. The following two operations are performed iteratively for
each sentence (pair).

\begin{itemize}

\item Exclude the previous expected counts of the current sentence (pair) from the model, and then derive the current sentence in all possible ways, calculating the new expected counts for the words (see Section~\ref{sec:expectation}), that is, we calculate the expected probabilities of the $\mathcal{F}_k^{k'}$ being words given the data excluding $\mathcal{F}$, i.e. $\mathbf{E}_{\mathbb{F}/\{\mathcal{F}\}}( P(\mathcal{F}_k^{k'}|\mathcal{F}))= P(\mathcal{F}_k^{k'}|\mathcal{F},\mathcal{M})$ in a similar manner
to the marginalization in the Gibbs sampling process which we are replacing;

\item Update the respective model $\mathcal{M}$ or $\mathcal{B}$ according to these expectations (see Section\ref{sec:maximization}).

\end{itemize}

\subsection{Expectation}
\label{sec:expectation}
\subsubsection{Monolingual Expectation}

$P(\mathcal{F}_k^{k'}|\mathcal{F},\mathcal{M})$ is the marginal probability of all the possible $F \in \mathcal{F}$ that contain $\mathcal{F}_k^{k'}$ as a word, 
which can be calculated efficiently through dynamic programming (the process is similar to the foreward-backward algorithm in training a hidden Markov model (HMM)~\cite{rabiner1989tutorial}):
\begin{align}
&\quad P_a(k) =\sum_{u=1}^U P_a(k-u) P_\mathcal{M}(\mathcal{F}_{k-u}^k) \nonumber \\
&\quad P_b(k') =\sum_{u=1}^U P_b(k'+u) P_\mathcal{M}(\mathcal{F}_{k'}^{k'+u}) \nonumber \\
&P(\mathcal{F}_k^{k'}|\mathcal{F},\mathcal{M})  =  P_a(k) P_\mathcal{M}(\mathcal{F}_k^{k'}) P_b(k'), \label{eq:mono:dp} 
\end{align}
where $U$ is the predefined maximum length of foreign language words, $P_a(k)$ and $P_b(k')$ are the forward and backward probabilities, respectively. 
This section uses a unigram model for description convenience, but the method can be extended to $n$-gram models.

\subsubsection{Bilingual Expectation}

$P(\mathcal{F}_k^{k'}|\mathcal{F},E,\mathcal{B})$  is the marginal probability of all the possible $F \in \mathcal{F}$ that contain $\mathcal{F}_k^{k'}$ as a word and are aligned with $E$, formulated as:
\begin{align}
&P(\mathcal{F}_k^{k'}|\mathcal{F},E,\mathcal{B})=\sum_{\substack{F\in \mathcal{F}\\  \mathcal{F}_k^{k'} \in F} }\sum_a P(F,a|E,\mathcal{B})  \nonumber \\
&\quad \approx  \sum_{\substack{F\in \mathcal{F}\\F_{j_k} = \mathcal{F}_k^{k'}} } \sum_a \prod_{j=1}^{J}P(a_j|j,I,J)P_\mathcal{B}(f_j|e_{a_j}) \nonumber \\
&\quad= \sum_{\substack{F\in \mathcal{F}\\f_{j_k} = \mathcal{F}_k^{k'}} } \prod_{j=1}^{J} \sum_a P(a_j|j,I,J)P_\mathcal{B}(f_j|e_{a_j}), \label{eq:biling:expectation}
\end{align}
where $J$ and $I$ are the number of foreign and English words, respectively, and $a_j$ is the position of the English word that is aligned to $f_j$ in the alignment $a$.  For the alignment we employ an approximation to IBM model 2~\cite{brown1993mathematics,och03:asc} described below.

We define the conditional probability of $f_j$ given the corresponding English sentence $E$ and the model $\mathcal{B}$ as:
\begin{align}
P_\mathcal{B}(f_j|E)&=\sum_a P(a_j|j,I,J) P_\mathcal{B}(f_j|e_{a_j}) \label{eq:biling:fe}  
\end{align}

Then, the previous dynamic programming method can be extended to the bilingual expectation
\begin{align}
&P_a(k|E) = \sum_{u=1}^U P_a(k-u|E) P_\mathcal{B}(\mathcal{F}_{k-u}^k|E) \nonumber \\
&P_b(k'|E)=\sum_{u=1}^U P_b(k'+u|E) P_\mathcal{B}(\mathcal{F}_{k'}^{k'+u}|E)\nonumber \\
&P(F_k^{k'}|\mathcal{F},E,\mathcal{B}) =  P_a(k|E) P_\mathcal{B}(\mathcal{F}_k^{k'}|E) P_b(k'|E). \label{eq:biling:dp} 
\end{align}

Eq.~\ref{eq:biling:fe} can be rewritten (as in IBM model 2):
\begin{align}
P_\mathcal{B}(f_j|E)&=\sum_{i=1}^I P^*(i|j,I,J)  P_\mathcal{B}(f_j|e_i)   \label{eq:biling:fe2} \\
P^*(i|j,I,J)&=\sum_{a:a_j=i}P(a_j|,j,I,J) \nonumber
\end{align}
In order to maintain both speed and accuracy, the following window function is adopted 
\begin{align}
&P^*(i|j,I,J) \approx P^*(i|k,I,K) = \nonumber \\ 
& \, \left\{
\begin{array}{ll}
e^{-|i-kI/K|}/\sigma  & |i-kI/K| \leqslant \delta_b/2 \\
\lambda_\phi           & e_i\textrm{ is empty word} \\
0                      & \textrm{otherwise} \\
\end{array}
\right. \label{eq:biling:window}
\end{align}
where $K$ is the number of characters in $\mathcal{F}$, and  the $k$-th character
is the start of the word $f_{j}$,  since $j$ and $J$ are unknown during
the computation of dynamic programming. $\delta_b$ is the window size,
$\lambda_\phi$ is the prior probability of an empty English word,  and
$\sigma$ ensures all the items sum to 1.


\subsection{Maximization}
\label{sec:maximization}

Inspired by \cite{teh2006hierarchical,mochihashi2009bayesian,neubig2010learning,teh2010hierarchical}, we employ a Pitman-Yor process model to build
the segmentation model $\mathcal{M}$ or $\mathcal{B}$.  The monolingual
model $\mathcal{M}$ is
\begin{align}
& P_\mathcal{M}(f_j) =   \nonumber \\
&\quad\quad \frac{ \mathrm{max}\big(n\small(f_j\small)-d,0\big) + (\theta +d \cdot n_\mathcal{M})G_0\small(f_j\small) } {\sum_{f_j'}n\small(f_j'\small)+\theta}  \nonumber \\
&\quad n_\mathcal{M}=\big|\{f_j \, | \, n(f_j) \geqslant d\}\big| ,  \label{eq:dirichlet:mono} 
\end{align}
where $f_j$ is a foreign language word, and $n(f_j)$ is the observed counts of $f_j$, $\theta$ is named the strength parameter, $G_0(f_j)$ is named the base distribution of $f_j$, and $d$ is the discount. 

 The bilingual model is
\begin{align}
&P_\mathcal{B}(f_j|e_i)   =  \nonumber \\
&\quad \frac{ \mathrm{max}\big(n\small(f_j,e_i\small)-d,0\big) + (\theta +d \cdot n_{e_i})G_0\small(f_j|e_i\small) } {\sum_{f_j'}n\small(f_j',e_i\small)+\theta} \nonumber \\
&\qquad \quad n_{e_i}=\big|\{x \, | \,  n(x,e_i) \geqslant d\}\big|.  \label{eq:dirichlet:biling}  
\end{align}

In Eqs.~\ref{eq:dirichlet:mono} and \ref{eq:dirichlet:biling},
\begin{align}
&\quad \, n(f_j) = \sum_{\substack{\mathcal{F} \in \mathbb{F}} }P(f_j|\mathcal{F},\mathcal{M}) \label{eq:nf}\\
&n(f_j,e_i)=   \nonumber \\
&\sum_{\substack{(\mathcal{F},E) \in \mathbb{B}}}P(f_j|\mathcal{F},E,\mathcal{B})  \frac{P^*(i|j,I,J)  P_\mathcal{B}(f_j|e_i)}{\sum_{i'=1}^I P^*(i'|j,I,J)  P_\mathcal{B}(f_j|e_{i'})}.
\end{align}

\section{Complexity Analysis}
\label{sec:complexity}
The computational complexity of our method is linear in the number of
iterations, the size of the corpus, and the complexity of calculating
the expectations on each sentence or sentence pair. In practical
applications, the size of the corpus is fixed, and we found empirically that the number of 
iterations required by the proposed method for convergence is usually small (less than five
iterations). We now look in more detail at the complexity of the expectation calculation in
monolingual and bilingual models.

The monolingual expectation is calculated according to Eq.~\ref{eq:mono:dp}; the
complexity is linear in the length of sentences and the square of the
predefined maximum length of words. Thus its overall complexity is
\begin{equation}
O_{\textrm{monoling}}^{\textrm{unigram}}=O(N_{\textrm{i}} |\mathbb{F}| K U^2), \label{eq:complexity:monoUnigram}
\end{equation}
where $N_{\textrm{i}}$ is the number of iterations, $K$ is the average number of characters per sentence, and $U$ is the predefined maximum length of words.

For the monolingual bigram model, the number of states in the HMM  is
$U$ times  more than  that of the monolingual unigram model, as the
states at specific position of $F$ are not only related to the length of
the current word, but also related to the length of the word before it.
Thus its complexity is $U^2$ times the unigram model's complexity:
\begin{equation}
O_{\textrm{monoling}}^{\textrm{bigram}}=O(N_{\textrm{i}} |\mathbb{F}| K  U^4). \label{eq:complexity:monoBigram}
\end{equation}

The bilingual expectation  is given by Eq.~\ref{eq:biling:dp}, whose
complexity is the same as the monolingual case. However, the
complexity of calculating the transition probability, in
Eqs.~\ref{eq:biling:fe2} and \ref{eq:biling:window}, is $O(\delta_b)$.
Thus its overall complexity is:
\begin{equation}
O_{\textrm{biling}}^{\textrm{unigram}}=O(N_{\textrm{i}} |\mathbb{F}| K U^2 \delta_b). \label{eq:complexity:bilingUnigram}
\end{equation}

\section{Experiments}
In this section, the proposed method is first validated on
monolingual segmentation tasks, and then evaluated in the context of SMT to study whether
the translation quality, measured by BLEU, can be improved.

\subsection{Experimental Settings}

\subsubsection{Experimental Corpora}
Two monolingual corpora and two bilingual corpora are
used (Table~\ref{tab:dataset}). CHILDES~\cite{macwhinney1985child} 
is the most common test corpus for UWS methods. 
The SIGHAN-MSR corpus~\cite{emerson2005second} 
consists of manually segmented simplified Chinese news text, released in the SIGHAN bakeoff
2005 shared tasks. 

The first bilingual corpus: OpenMT06 was used in the NIST open machine translation 2006
Evaluation~\footnote{\url{http://www.itl.nist.gov/iad/mig//tests/mt/2006/}}. 
We removed the United Nations corpus and the traditional Chinese
data sets from the constraint training resources. The data sets of NIST
Eval 2002 to 2005 were used as the development for MERT
tuning~\cite{och2003minimum}.  This data set mainly consists of news
text~\footnote{It also contains a small number of web blogs}.  PatentMT9
is from the shared task of NTCIR-9 patent machine translation
. The training set consists of 1 million parallel sentences extracted from
patent documents, and the development set and test set both consist of
2000 sentences.

\begin{table}
{\small
\begin{center}
\begin{tabular}{|lrrr|}
\hline
\bf Corpus & \bf Type & \bf \# Sentences & \bf \# Characters  \\ 
\hline
CHILDES&   Mono. & 9,790 & 95,809\\
SIGHAN-MSR &  Mono. & 90,903 & 4,234,824 \\
\hline
OpenMT06 & Biling.  & 437,004 & 19,692,605 \\ 
PatentMT9 & Biling. & 1,004,000 & 63,130,757 \\
\hline
\end{tabular}
\end{center}
}
\caption{\label{tab:dataset} Experimental Corpora}
\end{table}

\subsubsection{Performance Measurement and Baseline Methods}
For the monolingual tasks, the F$_1$ score against the gold annotation
is adopted to measure the accuracy.  The results reported in related
papers are listed for comparison.

For the bilingual tasks, the publicly available system of
Moses~\cite{koehn2007moses} with default settings is employed to perform
machine translation, and BLEU~\cite{papineni2002bleu} was used to
evaluate the quality. Character-based segmentation, LDC segmenter and
Stanford Chinese segmenters were used as the baseline methods.

\subsubsection{Parameter settings}
The parameters are tuned on held-out data sets.  The maximum length of foreign language words is set to 4.  For the PYP model,  the base distribution adopts the formula  in \cite{chung2009unsupervised}, and the strength parameter is set to $1.0$, and the discount is set to $1.0 \times 10^{-6}$.


For bilingual segmentation,the
size of the alignment window is set to $6$; the 
probability $\lambda_\phi$ of foreign language words being generated by an empty
English word, was set to $0.3$.  

The training was started from assuming that there was no previous segmentations on each sentence (pair), and the number of iterations was fixed.  It was set to 3 for the monolingual unigram model, and 2 for the bilingual unigram model,  which provided slightly higher BLEU scores on the development set than the other settings.  The monolingual bigram model, however, was slower to converge, so we started it from the segmentations of the unigram model, and using 10 iterations.

\subsection{Monolingual Segmentation Results}

In monolingual segmentation, the proposed methods with both unigram and
bigram models were tested. Experimental results show that they are
competitive to state-of-the-art baselines in both accuracy and speed
(Table~\ref{tab:mono:acc}). Note that the comparison of speed is only
for reference because the times are obtained
from their respective papers.

\begin{table}{
\small
\begin{center}
\begin{threeparttable}
\begin{tabular}{|l|cc|rr|}
\hline
\bf Method &\multicolumn{2}{|c|}{\bf Accuracy} &\multicolumn{2}{|c|}{\bf Time} \\
 &\bf CHILD. & \bf MSR      &\bf CHILD. &\bf MSR \\
\hline
NPY(bigram)\tnote{a}  & 0.750       & 0.802     &17 m   & -- \\
NPY(trigram)\tnote{a} & 0.757       &{\bf 0.807}& --    & -- \\
HDP(bigram)\tnote{b}  & 0.723       & --        & 10 h  & --\\
Fitness\tnote{c}      & --          & 0.667     & --    & --     \\
\hline 
Prop.(unigram)          & 0.729       & 0.804   &       3 s & 50 s\\
Prop.(bigram)           & {\bf 0.774} & 0.806   &      15 s & 2530 s \\
\hline
\end{tabular}
\begin{tablenotes}
\item[a] by (Mochihashi et al.,2009);
\item[b] by (Goldwater et al.,2009);
\item[c] by (Zhao and Kit, 2008).
\end{tablenotes}
\end{threeparttable}
\end{center}
\caption{\label{tab:mono:acc} Results on Monolingual Corpora.}
}\end{table}

\subsection{Bilingual Segmentation Results}

Table~\ref{tab:biling:acc} presents the BLEU scores for Moses using
different segmentation methods. Each experiment was performed three
times.  The proposed method with monolingual bigram model performed poorly on the
Chinese monolingual segmentation task; thus, it was not tested. We intended to test \cite{mochihashi2009bayesian}, but found it impracticable on large-scale corpora.

The experimental results show that the proposed UWS methods are comparable to the Stanford segmenters on the OpenMT06 corpus, while achieves a 0.96 BLEU increase on the PatentMT9 corpus. This is because this corpus is out-of-domain for the  supervised segmenters. The CTB and PKU Stanford segmenter were both trained on annotated news text, which was the major domain of OpenMT06. 


\begin{table}{\small
\begin{center}
\begin{tabular}{|l|cc|}
\hline
\bf Method &\multicolumn{2}{|c|}{\bf BLEU} \\
 &\bf OpenMT06 & \bf PatentMT9     \\
 \hline 
Character                   & 29.50 $\pm$ 0.03 & 28.36 $\pm$ 0.09 \\
LDC                         & 31.33 $\pm$ 0.10 & 30.22 $\pm$ 0.14 \\
Stanford(CTB)               & {\bf 31.68 $\pm$ 0.25} & 30.77 $\pm$ 0.13\\ 
Stanford(PKU)               & 31.54 $\pm$ 0.13 & 30.86 $\pm$ 0.04  \\
\hline
Prop.(mono.)                    & 31.47 $\pm$ 0.18 & 31.62 $\pm$ 0.06  \\    
Prop.(biling.)               &  31.61 $\pm$ 0.14 &{\bf  31.73 $\pm$ 0.05}    \\ 
\hline
\end{tabular}
\end{center}
\caption{\label{tab:biling:acc} Results on Bilingual Corpora.}
}\end{table}

Table~\ref{tab:biling:time} presents the run times of the proposed methods on
the bilingual corpora.  The program is single threaded and implemented in C++. The time cost of the bilingual models is about
 5 times that of the monolingual model, which is consistent with
the complexity analysis in Section~\ref{sec:complexity}.

\begin{table}{\small
\begin{center}
\begin{tabular}{|l|rr|rr|}
\hline
\bf Method&\multicolumn{2}{|c|}{\bf Time} \\
 &\bf OpenMT06 & \bf PatentMT9  \\
 \hline 
Prop.(mono.)                      &      28 m & 1 h 01 m \\    
Prop.(biling.)        &  2 h 25 m & 5 h 02 m \\  
\hline
\end{tabular}
\end{center}
\caption{\label{tab:biling:time} Time Costs on Bilingual Corpora.}
}\end{table}

\section{Conclusion}
This paper is devoted to large-scale Chinese UWS for SMT.  An efficient
unified monolingual and bilingual UWS method is proposed and applied to
large-scale bilingual corpora.

Complexity analysis shows that
our method is capable of scaling to large-scale corpora. 
This was verified by experiments on a corpus of 1-million sentence pairs on
which traditional MCMC approaches would struggle \cite{xu2008bayesian}.

The proposed method does not require any annotated data, but the SMT system
with it can achieve comparable performance compared to
state-of-the-art supervised word segmenters trained on precious annotated
data.  Moreover, the proposed method yields 0.96 BLEU improvement relative to
supervised word segmenters on an out-of-domain corpus.  Thus, we believe that the
proposed method would benefit SMT related to low-resource languages where annotated
data are scare, and would also find application in domains that differ too greatly
from the domains on which supervised word segmenters were trained.

In future research, we plan to improve the bilingual UWS through applying VB and integrating more accurate alignment models such as HMM models and IBM model 4. 

%\section*{Acknowledgments}

\bibliographystyle{acl}
% you bib file should really go here
\bibliography{segmentRef}


\end{document}

